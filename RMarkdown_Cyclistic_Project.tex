% Options for packages loaded elsewhere
\PassOptionsToPackage{unicode}{hyperref}
\PassOptionsToPackage{hyphens}{url}
%
\documentclass[
]{article}
\usepackage{amsmath,amssymb}
\usepackage{iftex}
\ifPDFTeX
  \usepackage[T1]{fontenc}
  \usepackage[utf8]{inputenc}
  \usepackage{textcomp} % provide euro and other symbols
\else % if luatex or xetex
  \usepackage{unicode-math} % this also loads fontspec
  \defaultfontfeatures{Scale=MatchLowercase}
  \defaultfontfeatures[\rmfamily]{Ligatures=TeX,Scale=1}
\fi
\usepackage{lmodern}
\ifPDFTeX\else
  % xetex/luatex font selection
\fi
% Use upquote if available, for straight quotes in verbatim environments
\IfFileExists{upquote.sty}{\usepackage{upquote}}{}
\IfFileExists{microtype.sty}{% use microtype if available
  \usepackage[]{microtype}
  \UseMicrotypeSet[protrusion]{basicmath} % disable protrusion for tt fonts
}{}
\makeatletter
\@ifundefined{KOMAClassName}{% if non-KOMA class
  \IfFileExists{parskip.sty}{%
    \usepackage{parskip}
  }{% else
    \setlength{\parindent}{0pt}
    \setlength{\parskip}{6pt plus 2pt minus 1pt}}
}{% if KOMA class
  \KOMAoptions{parskip=half}}
\makeatother
\usepackage{xcolor}
\usepackage[margin=1in]{geometry}
\usepackage{color}
\usepackage{fancyvrb}
\newcommand{\VerbBar}{|}
\newcommand{\VERB}{\Verb[commandchars=\\\{\}]}
\DefineVerbatimEnvironment{Highlighting}{Verbatim}{commandchars=\\\{\}}
% Add ',fontsize=\small' for more characters per line
\usepackage{framed}
\definecolor{shadecolor}{RGB}{248,248,248}
\newenvironment{Shaded}{\begin{snugshade}}{\end{snugshade}}
\newcommand{\AlertTok}[1]{\textcolor[rgb]{0.94,0.16,0.16}{#1}}
\newcommand{\AnnotationTok}[1]{\textcolor[rgb]{0.56,0.35,0.01}{\textbf{\textit{#1}}}}
\newcommand{\AttributeTok}[1]{\textcolor[rgb]{0.13,0.29,0.53}{#1}}
\newcommand{\BaseNTok}[1]{\textcolor[rgb]{0.00,0.00,0.81}{#1}}
\newcommand{\BuiltInTok}[1]{#1}
\newcommand{\CharTok}[1]{\textcolor[rgb]{0.31,0.60,0.02}{#1}}
\newcommand{\CommentTok}[1]{\textcolor[rgb]{0.56,0.35,0.01}{\textit{#1}}}
\newcommand{\CommentVarTok}[1]{\textcolor[rgb]{0.56,0.35,0.01}{\textbf{\textit{#1}}}}
\newcommand{\ConstantTok}[1]{\textcolor[rgb]{0.56,0.35,0.01}{#1}}
\newcommand{\ControlFlowTok}[1]{\textcolor[rgb]{0.13,0.29,0.53}{\textbf{#1}}}
\newcommand{\DataTypeTok}[1]{\textcolor[rgb]{0.13,0.29,0.53}{#1}}
\newcommand{\DecValTok}[1]{\textcolor[rgb]{0.00,0.00,0.81}{#1}}
\newcommand{\DocumentationTok}[1]{\textcolor[rgb]{0.56,0.35,0.01}{\textbf{\textit{#1}}}}
\newcommand{\ErrorTok}[1]{\textcolor[rgb]{0.64,0.00,0.00}{\textbf{#1}}}
\newcommand{\ExtensionTok}[1]{#1}
\newcommand{\FloatTok}[1]{\textcolor[rgb]{0.00,0.00,0.81}{#1}}
\newcommand{\FunctionTok}[1]{\textcolor[rgb]{0.13,0.29,0.53}{\textbf{#1}}}
\newcommand{\ImportTok}[1]{#1}
\newcommand{\InformationTok}[1]{\textcolor[rgb]{0.56,0.35,0.01}{\textbf{\textit{#1}}}}
\newcommand{\KeywordTok}[1]{\textcolor[rgb]{0.13,0.29,0.53}{\textbf{#1}}}
\newcommand{\NormalTok}[1]{#1}
\newcommand{\OperatorTok}[1]{\textcolor[rgb]{0.81,0.36,0.00}{\textbf{#1}}}
\newcommand{\OtherTok}[1]{\textcolor[rgb]{0.56,0.35,0.01}{#1}}
\newcommand{\PreprocessorTok}[1]{\textcolor[rgb]{0.56,0.35,0.01}{\textit{#1}}}
\newcommand{\RegionMarkerTok}[1]{#1}
\newcommand{\SpecialCharTok}[1]{\textcolor[rgb]{0.81,0.36,0.00}{\textbf{#1}}}
\newcommand{\SpecialStringTok}[1]{\textcolor[rgb]{0.31,0.60,0.02}{#1}}
\newcommand{\StringTok}[1]{\textcolor[rgb]{0.31,0.60,0.02}{#1}}
\newcommand{\VariableTok}[1]{\textcolor[rgb]{0.00,0.00,0.00}{#1}}
\newcommand{\VerbatimStringTok}[1]{\textcolor[rgb]{0.31,0.60,0.02}{#1}}
\newcommand{\WarningTok}[1]{\textcolor[rgb]{0.56,0.35,0.01}{\textbf{\textit{#1}}}}
\usepackage{graphicx}
\makeatletter
\def\maxwidth{\ifdim\Gin@nat@width>\linewidth\linewidth\else\Gin@nat@width\fi}
\def\maxheight{\ifdim\Gin@nat@height>\textheight\textheight\else\Gin@nat@height\fi}
\makeatother
% Scale images if necessary, so that they will not overflow the page
% margins by default, and it is still possible to overwrite the defaults
% using explicit options in \includegraphics[width, height, ...]{}
\setkeys{Gin}{width=\maxwidth,height=\maxheight,keepaspectratio}
% Set default figure placement to htbp
\makeatletter
\def\fps@figure{htbp}
\makeatother
\setlength{\emergencystretch}{3em} % prevent overfull lines
\providecommand{\tightlist}{%
  \setlength{\itemsep}{0pt}\setlength{\parskip}{0pt}}
\setcounter{secnumdepth}{-\maxdimen} % remove section numbering
\ifLuaTeX
  \usepackage{selnolig}  % disable illegal ligatures
\fi
\IfFileExists{bookmark.sty}{\usepackage{bookmark}}{\usepackage{hyperref}}
\IfFileExists{xurl.sty}{\usepackage{xurl}}{} % add URL line breaks if available
\urlstyle{same}
\hypersetup{
  pdftitle={Cyclistic Case Study},
  pdfauthor={Drew Radovich},
  hidelinks,
  pdfcreator={LaTeX via pandoc}}

\title{Cyclistic Case Study}
\author{Drew Radovich}
\date{2024-02-26}

\begin{document}
\maketitle

\hypertarget{this-is-an-rmarkdown-file-to-walk-through-the-google-data-analyst-capstone-project-cyclistic-case-study.-the-data-for-this-study-has-been-made-available-by-motivate-international-inc.-under-this-license.}{%
\subsubsection{\texorpdfstring{\textbf{This is an RMarkdown file to walk
through the Google Data Analyst Capstone Project: Cyclistic Case Study.
The data for this study has been made available by Motivate
International Inc.~under this
\href{https://divvy-tripdata.s3.amazonaws.com/index.html}{license}.}}{This is an RMarkdown file to walk through the Google Data Analyst Capstone Project: Cyclistic Case Study. The data for this study has been made available by Motivate International Inc.~under this license.}}\label{this-is-an-rmarkdown-file-to-walk-through-the-google-data-analyst-capstone-project-cyclistic-case-study.-the-data-for-this-study-has-been-made-available-by-motivate-international-inc.-under-this-license.}}

Cyclistic is a bike-share company based in Chicago that provides over
5,800 bikes at 600 docking stations. To appeal to those who may not be
able to use conventional two-wheel bicycles, Cyclistic offers hand
tricycles, reclining bikes, and cargo bikes in addition to traditional
bicycles. These services are used mostly by casual app users, but also
by annual members.

For this case study, you are a junior data analyst who is part of the
Cyclistic marketing analysts team, tasked with understanding the
difference between casual and member use of Cyclistic bike shares. From
this, your team will initiate a new marketing campaign aimed at
converting casual users to annual members.

\hypertarget{key-stakeholders-of-this-project-include}{%
\subsection{Key stakeholders of this project
include:}\label{key-stakeholders-of-this-project-include}}

\textbf{Lily Moreno:} your direct supervisor and manager of marketing.
Moreno is in charge of developing campaigns and initiatives to promote
use of the bike-share program.

\textbf{Cyclistic marketing analytics team:} you are part of this team
who is responsible for collecting, analyzing, and reporting Cyclisitc
ride-share data to inform marketing campaign recommendations.

\textbf{Cyclistic executive team:} they will approve or deny any
recommended marketing program

\hypertarget{business-task}{%
\subsection{Business Task:}\label{business-task}}

To analyze ride-share data and define key characteristics and
distinctions in bike use for casual users and annual members.

In order to answer this business question, I will be using RStudio and
the R programming language. I have chosen R because of it's ability to
handle large data sets and it's ability to transform data and make
visualizations. I have also chosen to use Tableau in order to better
visualize my findings and condense information into a more
comprehensible visual for stakeholders. To help my analysis in R, I will
be utilizing the \emph{tidyverse}, \emph{janitor}, and \emph{lubridate}
packages.

\begin{Shaded}
\begin{Highlighting}[]
\DocumentationTok{\#\# Load packages for use}
\FunctionTok{library}\NormalTok{(tidyverse)}
\FunctionTok{library}\NormalTok{(janitor)}
\FunctionTok{library}\NormalTok{(lubridate)}
\end{Highlighting}
\end{Shaded}

To start analyzing, I downloaded the 2023 ride-share data here. Since
these files are in .csv format, I can add them to my R environment with
the \emph{read.csv()} function. This is also an opportunity to clarify
the names of the files to aid in analysis.

\begin{Shaded}
\begin{Highlighting}[]
\DocumentationTok{\#\#Adding a clear name for data \textless{}{-} Importing .csv file}

\NormalTok{Jan2023 }\OtherTok{\textless{}{-}} \FunctionTok{read.csv}\NormalTok{(}\StringTok{"2023\_trip\_data/2023\_01.csv"}\NormalTok{)}
\NormalTok{Feb2023 }\OtherTok{\textless{}{-}} \FunctionTok{read.csv}\NormalTok{(}\StringTok{"2023\_trip\_data/2023\_02.csv"}\NormalTok{)}
\NormalTok{Mar2023 }\OtherTok{\textless{}{-}} \FunctionTok{read.csv}\NormalTok{(}\StringTok{"2023\_trip\_data/2023\_03.csv"}\NormalTok{)}
\NormalTok{Apr2023 }\OtherTok{\textless{}{-}} \FunctionTok{read.csv}\NormalTok{(}\StringTok{"2023\_trip\_data/2023\_04.csv"}\NormalTok{)}
\NormalTok{May2023 }\OtherTok{\textless{}{-}} \FunctionTok{read.csv}\NormalTok{(}\StringTok{"2023\_trip\_data/2023\_05.csv"}\NormalTok{)}
\NormalTok{Jun2023 }\OtherTok{\textless{}{-}} \FunctionTok{read.csv}\NormalTok{(}\StringTok{"2023\_trip\_data/2023\_06.csv"}\NormalTok{)}
\NormalTok{Jul2023 }\OtherTok{\textless{}{-}} \FunctionTok{read.csv}\NormalTok{(}\StringTok{"2023\_trip\_data/2023\_07.csv"}\NormalTok{)}
\NormalTok{Aug2023 }\OtherTok{\textless{}{-}} \FunctionTok{read.csv}\NormalTok{(}\StringTok{"2023\_trip\_data/2023\_08.csv"}\NormalTok{)}
\NormalTok{Sep2023 }\OtherTok{\textless{}{-}} \FunctionTok{read.csv}\NormalTok{(}\StringTok{"2023\_trip\_data/2023\_09.csv"}\NormalTok{)}
\NormalTok{Oct2023 }\OtherTok{\textless{}{-}} \FunctionTok{read.csv}\NormalTok{(}\StringTok{"2023\_trip\_data/2023\_10.csv"}\NormalTok{)}
\NormalTok{Nov2023 }\OtherTok{\textless{}{-}} \FunctionTok{read.csv}\NormalTok{(}\StringTok{"2023\_trip\_data/2023\_11.csv"}\NormalTok{)}
\NormalTok{Dec2023 }\OtherTok{\textless{}{-}} \FunctionTok{read.csv}\NormalTok{(}\StringTok{"2023\_trip\_data/2023\_12.csv"}\NormalTok{)}
\end{Highlighting}
\end{Shaded}

To make sure the data is reliable and can be analyzed as a whole, I want
to check that each table has identical column names and data types. This
can be done with the \emph{str()} function.

\begin{Shaded}
\begin{Highlighting}[]
\DocumentationTok{\#\# Checking each file for compatibility to merge into one data set}
\FunctionTok{str}\NormalTok{(Jan2023)}
\FunctionTok{str}\NormalTok{(Feb2023)}
\FunctionTok{str}\NormalTok{(Mar2023)}
\FunctionTok{str}\NormalTok{(Apr2023)}
\FunctionTok{str}\NormalTok{(May2023)}
\FunctionTok{str}\NormalTok{(Jun2023)}
\FunctionTok{str}\NormalTok{(Jul2023)}
\FunctionTok{str}\NormalTok{(Aug2023)}
\FunctionTok{str}\NormalTok{(Sep2023)}
\FunctionTok{str}\NormalTok{(Oct2023)}
\FunctionTok{str}\NormalTok{(Nov2023)}
\FunctionTok{str}\NormalTok{(Dec2023)}
\end{Highlighting}
\end{Shaded}

Since each table contain the same data columns and types, they can be
merged into one data drame.

\begin{Shaded}
\begin{Highlighting}[]
\DocumentationTok{\#\# Create new dataframe using bind\_rows}
\NormalTok{merged\_trips }\OtherTok{\textless{}{-}} \FunctionTok{bind\_rows}\NormalTok{(Jan2023, Feb2023, Mar2023, Apr2023, May2023, Jun2023, Jul2023, Aug2023, Sep2023, Oct2023, Nov2023, Dec2023)}
\end{Highlighting}
\end{Shaded}

Now that all of the data is in one data frame, we can begin to ensure
that our data is clean by removing duplicates and verifying or removing
empty data points. It is important to clean data and make sure that data
types are equivalent so that further analysis is as accurate as
possible.

\begin{Shaded}
\begin{Highlighting}[]
\DocumentationTok{\#\# Clean data frame by removing spaces, parentheses, camelCase, etc.}
\NormalTok{merged\_trips }\OtherTok{\textless{}{-}} \FunctionTok{clean\_names}\NormalTok{(merged\_trips)}

\DocumentationTok{\#\# Remove empty columns and rows}
\FunctionTok{remove\_empty}\NormalTok{(merged\_trips, }\AttributeTok{which =} \FunctionTok{c}\NormalTok{())}
\end{Highlighting}
\end{Shaded}

In order to better understand ride-share usage, I want to look at usage
on the hourly, daily, and monthly scale. I believe this will provide
better insights to how casual and annual members use the ride-share
program.

\begin{Shaded}
\begin{Highlighting}[]
\DocumentationTok{\#\# Calculate day of the week using wday()}
\NormalTok{merged\_trips}\SpecialCharTok{$}\NormalTok{day\_of\_week}\OtherTok{\textless{}{-}} \FunctionTok{wday}\NormalTok{(merged\_trips}\SpecialCharTok{$}\NormalTok{started\_at, }\ConstantTok{TRUE}\NormalTok{, }\ConstantTok{TRUE}\NormalTok{)}

\DocumentationTok{\#\# Calculate the start hour for trips using POSIXct}
\NormalTok{merged\_trips}\SpecialCharTok{$}\NormalTok{start\_hour }\OtherTok{\textless{}{-}} \FunctionTok{format}\NormalTok{(}\FunctionTok{as.POSIXct}\NormalTok{(merged\_trips}\SpecialCharTok{$}\NormalTok{started\_at), }\StringTok{"\%H"}\NormalTok{)}

\DocumentationTok{\#\# Extract Month}
\NormalTok{merged\_trips}\SpecialCharTok{$}\NormalTok{month }\OtherTok{\textless{}{-}} \FunctionTok{month}\NormalTok{(merged\_trips}\SpecialCharTok{$}\NormalTok{started\_at, }\AttributeTok{label =} \ConstantTok{TRUE}\NormalTok{, }\AttributeTok{abbr =} \ConstantTok{TRUE}\NormalTok{)}
\end{Highlighting}
\end{Shaded}

Next, I will calculate time for each individual trip using the
start\_time and end\_time columns of my data frame. This will allow me
to further filter the data to eliminate trips with no duration that may
have been cancelled by users or failed to accurately report data.

Now that the cleaning is complete, I can start to explore the data and
analyze for insights.

First, I want to check how ride-share use varies throughout the week by
member type. I believe that annual members are more likely to use the
ride-share for commuting purposes, meaning that member trips should
outnumber and be more consistent on weekdays (Monday through Friday) and
casual user trips should outnumber annual members on the weekend
(Saturday and Sunday).

\begin{Shaded}
\begin{Highlighting}[]
\DocumentationTok{\#\# Plot number of rides by member type, options(scipen = ) is used to remove scientific value from results}
\FunctionTok{options}\NormalTok{(}\AttributeTok{scipen =} \DecValTok{999}\NormalTok{)}
\FunctionTok{ggplot}\NormalTok{(}\AttributeTok{data =}\NormalTok{ cleaned\_trips) }\SpecialCharTok{+}
\FunctionTok{aes}\NormalTok{(}\AttributeTok{x =}\NormalTok{ day\_of\_week, }\AttributeTok{fill =}\NormalTok{ member\_casual) }\SpecialCharTok{+}
\FunctionTok{geom\_bar}\NormalTok{(}\AttributeTok{position =} \StringTok{\textquotesingle{}dodge\textquotesingle{}}\NormalTok{) }\SpecialCharTok{+}
\FunctionTok{labs}\NormalTok{(}\AttributeTok{x =} \StringTok{\textquotesingle{}Day of Week\textquotesingle{}}\NormalTok{, }\AttributeTok{y =} \StringTok{\textquotesingle{}Number of Trips\textquotesingle{}}\NormalTok{, }\AttributeTok{fill =} \StringTok{\textquotesingle{}Member Type\textquotesingle{}}\NormalTok{, }\AttributeTok{title =} \StringTok{\textquotesingle{}Number of Rides per Day of Week\textquotesingle{}}\NormalTok{)}
\end{Highlighting}
\end{Shaded}

\includegraphics{RMarkdown_Cyclistic_Project_files/figure-latex/week day plot-1.pdf}

Now I want to check trips by month.

\begin{Shaded}
\begin{Highlighting}[]
\DocumentationTok{\#\# Plot number of trips by member type per month}
\FunctionTok{ggplot}\NormalTok{(}\AttributeTok{data =}\NormalTok{ cleaned\_trips) }\SpecialCharTok{+}
\FunctionTok{aes}\NormalTok{(}\AttributeTok{x =}\NormalTok{ month, }\AttributeTok{fill =}\NormalTok{ member\_casual) }\SpecialCharTok{+}
\FunctionTok{geom\_bar}\NormalTok{(}\AttributeTok{position =} \StringTok{\textquotesingle{}dodge\textquotesingle{}}\NormalTok{) }\SpecialCharTok{+}
\FunctionTok{labs}\NormalTok{(}\AttributeTok{x =} \StringTok{\textquotesingle{}Month\textquotesingle{}}\NormalTok{, }\AttributeTok{y =} \StringTok{\textquotesingle{}Number of Trips\textquotesingle{}}\NormalTok{, }\AttributeTok{fill =} \StringTok{\textquotesingle{}Member Type\textquotesingle{}}\NormalTok{, }\AttributeTok{title =} \StringTok{\textquotesingle{}Number of Trips per Month\textquotesingle{}}\NormalTok{)}
\end{Highlighting}
\end{Shaded}

\includegraphics{RMarkdown_Cyclistic_Project_files/figure-latex/month plot-1.pdf}

Next, I want to explore trips on a micro level and look at the starting
hours of each trip. This should further support my hypothesis about
annual members using Cyclistic bikes for commuting purposes. This will
be explained further in the analysis section of this project.

\begin{Shaded}
\begin{Highlighting}[]
\DocumentationTok{\#\# Plot trips per weekday by starting hour}
\FunctionTok{ggplot}\NormalTok{(}\AttributeTok{data =}\NormalTok{ cleaned\_trips) }\SpecialCharTok{+}
\FunctionTok{aes}\NormalTok{(}\AttributeTok{x =}\NormalTok{ start\_hour, }\AttributeTok{fill =}\NormalTok{ member\_casual) }\SpecialCharTok{+}
\FunctionTok{geom\_bar}\NormalTok{() }\SpecialCharTok{+}
\FunctionTok{facet\_wrap}\NormalTok{(}\SpecialCharTok{\textasciitilde{}}\NormalTok{day\_of\_week) }\SpecialCharTok{+}
\FunctionTok{labs}\NormalTok{(}\AttributeTok{x =} \StringTok{\textquotesingle{}Starting Hour\textquotesingle{}}\NormalTok{, }\AttributeTok{y =} \StringTok{\textquotesingle{}Number of Trips\textquotesingle{}}\NormalTok{, }\AttributeTok{fill =} \StringTok{\textquotesingle{}Member Type\textquotesingle{}}\NormalTok{, }\AttributeTok{title =} \StringTok{\textquotesingle{}Trips by Start Hour\textquotesingle{}}\NormalTok{) }\SpecialCharTok{+}
\FunctionTok{theme}\NormalTok{(}\AttributeTok{axis.text =} \FunctionTok{element\_text}\NormalTok{(}\AttributeTok{size =} \DecValTok{5}\NormalTok{))}
\end{Highlighting}
\end{Shaded}

\includegraphics{RMarkdown_Cyclistic_Project_files/figure-latex/start hour plot-1.pdf}

I also calculated the average trip duration for every day. This data can
be extremely helpful in understanding the purpose each member type
derives from using the ride-share.

\begin{Shaded}
\begin{Highlighting}[]
\DocumentationTok{\#\# Find the average trip duration }
\NormalTok{average\_trips }\OtherTok{\textless{}{-}} \FunctionTok{aggregate}\NormalTok{(cleaned\_trips}\SpecialCharTok{$}\NormalTok{trip\_duration, }\FunctionTok{list}\NormalTok{(cleaned\_trips}\SpecialCharTok{$}\NormalTok{member\_casual, cleaned\_trips}\SpecialCharTok{$}\NormalTok{day\_of\_week), }\AttributeTok{FUN =}\NormalTok{ mean)}
\end{Highlighting}
\end{Shaded}

Since aggregate() creates a table with placeholder column names, I want
to rename the columns to make it easier to understand and analyze.

\begin{Shaded}
\begin{Highlighting}[]
\DocumentationTok{\#\# Rename the column names to make analysis easier}
\FunctionTok{colnames}\NormalTok{(average\_trips) }\OtherTok{\textless{}{-}} \FunctionTok{c}\NormalTok{(}\StringTok{\textquotesingle{}member\_type\textquotesingle{}}\NormalTok{, }\StringTok{\textquotesingle{}day\_of\_week\textquotesingle{}}\NormalTok{, }\StringTok{\textquotesingle{}average\_duration\textquotesingle{}}\NormalTok{)}
\end{Highlighting}
\end{Shaded}

I want to plot this data using \emph{ggplot()}, but since the data type
\emph{} cannot be used, I will need to convert the data type of the
average\_duration column to \emph{}.

\begin{Shaded}
\begin{Highlighting}[]
\DocumentationTok{\#\# Convert data types for use in ggplot, since \textless{}difftime\textgreater{} cannot be used}
\NormalTok{average\_trips}\SpecialCharTok{$}\NormalTok{day\_of\_week }\OtherTok{\textless{}{-}} \FunctionTok{as.character}\NormalTok{(average\_trips}\SpecialCharTok{$}\NormalTok{day\_of\_week)}

\NormalTok{average\_trips}\SpecialCharTok{$}\NormalTok{average\_duration }\OtherTok{\textless{}{-}} \FunctionTok{as.numeric}\NormalTok{(average\_trips}\SpecialCharTok{$}\NormalTok{average\_duration, }\AttributeTok{units =} \StringTok{\textquotesingle{}secs\textquotesingle{}}\NormalTok{)}
\end{Highlighting}
\end{Shaded}

\begin{Shaded}
\begin{Highlighting}[]
\DocumentationTok{\#\# Create a plot showing the average trip duration throughout the week}
\FunctionTok{ggplot}\NormalTok{(average\_trips, }\FunctionTok{aes}\NormalTok{(}\AttributeTok{x =}\NormalTok{ average\_duration, }\AttributeTok{y =}\NormalTok{ day\_of\_week, }\AttributeTok{fill =}\NormalTok{ member\_type)) }\SpecialCharTok{+}
\FunctionTok{geom\_col}\NormalTok{(}\AttributeTok{position =} \StringTok{\textquotesingle{}dodge\textquotesingle{}}\NormalTok{) }\SpecialCharTok{+}
\FunctionTok{labs}\NormalTok{(}\AttributeTok{x =} \StringTok{\textquotesingle{}Average Duration\textquotesingle{}}\NormalTok{, }\AttributeTok{y =} \StringTok{\textquotesingle{}Day of Week\textquotesingle{}}\NormalTok{, }\AttributeTok{fill =} \StringTok{\textquotesingle{}Member Type\textquotesingle{}}\NormalTok{, }\AttributeTok{title =} \StringTok{\textquotesingle{}Average Trip Duration per Week Day\textquotesingle{}}\NormalTok{)}
\end{Highlighting}
\end{Shaded}

\includegraphics{RMarkdown_Cyclistic_Project_files/figure-latex/average trip plot-1.pdf}

Now that I have more insight into how Cyclistic bikes are used, I want
to explore where they are used. This information can help dictate which
locations should be targeted for marketing based on the number of trips
that occur.

\begin{Shaded}
\begin{Highlighting}[]
\DocumentationTok{\#\# Use descriptive analysis to find the most popular start and end stations for members and casual}

\DocumentationTok{\#\# Most popular end stations for members}
\NormalTok{member\_trips }\OtherTok{\textless{}{-}}\NormalTok{ cleaned\_trips[cleaned\_trips}\SpecialCharTok{$}\NormalTok{member\_casual }\SpecialCharTok{==} \StringTok{\textquotesingle{}member\textquotesingle{}}\NormalTok{,]}

\NormalTok{ end\_station\_counts }\OtherTok{\textless{}{-}}\NormalTok{ member\_trips}\SpecialCharTok{$}\NormalTok{end\_station\_name }\SpecialCharTok{\%\textgreater{}\%} \FunctionTok{table}\NormalTok{() }\SpecialCharTok{\%\textgreater{}\%}  \FunctionTok{sort}\NormalTok{(}\AttributeTok{decreasing =} \ConstantTok{TRUE}\NormalTok{)}

\FunctionTok{head}\NormalTok{(end\_station\_counts, }\DecValTok{10}\NormalTok{)}
\end{Highlighting}
\end{Shaded}

\begin{verbatim}
## .
##                              Clinton St & Washington Blvd 
##                       547259                        27446 
##     Kingsbury St & Kinzie St            Clark St & Elm St 
##                        26367                        24859 
##        Wells St & Concord Ln      Clinton St & Madison St 
##                        22249                        22096 
##            Wells St & Elm St     University Ave & 57th St 
##                        20227                        20217 
##         Broadway & Barry Ave       State St & Chicago Ave 
##                        19393                        19029
\end{verbatim}

\begin{Shaded}
\begin{Highlighting}[]
\DocumentationTok{\#\# Most popular start stations for members}
\NormalTok{member\_trips }\OtherTok{\textless{}{-}}\NormalTok{ cleaned\_trips[cleaned\_trips}\SpecialCharTok{$}\NormalTok{member\_casual }\SpecialCharTok{==} \StringTok{\textquotesingle{}member\textquotesingle{}}\NormalTok{,]}

\NormalTok{start\_station\_counts }\OtherTok{\textless{}{-}}\NormalTok{ member\_trips}\SpecialCharTok{$}\NormalTok{start\_station\_name }\SpecialCharTok{\%\textgreater{}\%} \FunctionTok{table}\NormalTok{() }\SpecialCharTok{\%\textgreater{}\%} \FunctionTok{sort}\NormalTok{(}\AttributeTok{decreasing =} \ConstantTok{TRUE}\NormalTok{)}

\FunctionTok{head}\NormalTok{(start\_station\_counts, }\DecValTok{10}\NormalTok{)}
\end{Highlighting}
\end{Shaded}

\begin{verbatim}
## .
##                              Clinton St & Washington Blvd 
##                       549587                        26216 
##     Kingsbury St & Kinzie St            Clark St & Elm St 
##                        26172                        25001 
##        Wells St & Concord Ln      Clinton St & Madison St 
##                        21419                        20596 
##            Wells St & Elm St     University Ave & 57th St 
##                        20400                        20038 
##         Broadway & Barry Ave     Loomis St & Lexington St 
##                        18959                        18901
\end{verbatim}

\begin{Shaded}
\begin{Highlighting}[]
\DocumentationTok{\#\# Most popular end stations for casual users}
\NormalTok{casual\_trips }\OtherTok{\textless{}{-}}\NormalTok{ cleaned\_trips[cleaned\_trips}\SpecialCharTok{$}\NormalTok{member\_casual }\SpecialCharTok{==} \StringTok{\textquotesingle{}casual\textquotesingle{}}\NormalTok{,]}

\NormalTok{end\_station\_counts2 }\OtherTok{\textless{}{-}}\NormalTok{ casual\_trips}\SpecialCharTok{$}\NormalTok{end\_station\_name }\SpecialCharTok{\%\textgreater{}\%} \FunctionTok{table}\NormalTok{() }\SpecialCharTok{\%\textgreater{}\%} \FunctionTok{sort}\NormalTok{(}\AttributeTok{decreasing =} \ConstantTok{TRUE}\NormalTok{)}

\FunctionTok{head}\NormalTok{(end\_station\_counts2, }\DecValTok{10}\NormalTok{)}
\end{Highlighting}
\end{Shaded}

\begin{verbatim}
## .
##                                               Streeter Dr & Grand Ave 
##                             381943                              49310 
##  DuSable Lake Shore Dr & Monroe St              Michigan Ave & Oak St 
##                              27539                              23688 
## DuSable Lake Shore Dr & North Blvd                    Millennium Park 
##                              23256                              22220 
##                Theater on the Lake                     Shedd Aquarium 
##                              17574                              15658 
##                     Dusable Harbor              Wells St & Concord Ln 
##                              13558                              11924
\end{verbatim}

\begin{Shaded}
\begin{Highlighting}[]
\DocumentationTok{\#\# Most popular start stations for casual users}
\NormalTok{casual\_trips }\OtherTok{\textless{}{-}}\NormalTok{ cleaned\_trips[cleaned\_trips}\SpecialCharTok{$}\NormalTok{member\_casual }\SpecialCharTok{==} \StringTok{\textquotesingle{}casual\textquotesingle{}}\NormalTok{,]}

\NormalTok{start\_station\_counts2 }\OtherTok{\textless{}{-}}\NormalTok{ casual\_trips}\SpecialCharTok{$}\NormalTok{start\_station\_name }\SpecialCharTok{\%\textgreater{}\%} \FunctionTok{table}\NormalTok{() }\SpecialCharTok{\%\textgreater{}\%} \FunctionTok{sort}\NormalTok{(}\AttributeTok{decreasing =} \ConstantTok{TRUE}\NormalTok{)}

\FunctionTok{head}\NormalTok{(start\_station\_counts2, }\DecValTok{10}\NormalTok{)}
\end{Highlighting}
\end{Shaded}

\begin{verbatim}
## .
##                                               Streeter Dr & Grand Ave 
##                             326129                              46030 
##  DuSable Lake Shore Dr & Monroe St              Michigan Ave & Oak St 
##                              30487                              22664 
## DuSable Lake Shore Dr & North Blvd                    Millennium Park 
##                              20338                              20228 
##                     Shedd Aquarium                Theater on the Lake 
##                              17783                              16359 
##                     Dusable Harbor              Wells St & Concord Ln 
##                              15491                              12171
\end{verbatim}

\hypertarget{analysis}{%
\subsection{Analysis}\label{analysis}}

From my exploration of the data, I have discovered a few distinguishing
features in the usage between casual users and annual members.Let's
explore these differences together:

First, by graphing out ride-share usage throughout the week, we can see
an immediate difference in how casual users and annual members ride. For
annual members, the graph appears like a typical bell curve with a
\textbf{higher volume of rides from Monday to Friday}, and the lowest
volume of the week falling on Saturday and Sunday. This supports my
hypothesis that \textbf{annual members are more likely to use ride-share
for commuting purposes}.

For casual users, the opposite is true. I found higher ride volumes on
weekend days (Friday, Saturday, Sunday), suggesting that
\textbf{ride-share is more recreational for casual users}.

These findings are further supported when we zoom out and look at ride
volume throughout the year. Ride volume for both members and casual
users increases during the summer months with \textbf{peaks in August
and July}, respectively. Since these are typically the warmest and
sunniest months, this makes sense for both people using bikes for
commuting and recreational purposes. The drastic \textbf{drop in volume
for the months of December through February} is also expected, given the
harshness of Chicago winters.

Looking at starting hours for rides throughout the week, there are
\textbf{peaks of activity at 8:00am and 5:00pm on weekdays}. This
activity is consistent for both casual users and annual members, meaning
that \emph{casual users may also use the ride-share app for commuting
during the week}, however not at the same volume as annual members. For
weekends, casual users and annual members also have similar behaviors.
Trips typically started later in the afternoon, with less noticeable
peaks, but a \textbf{more consistent usage in the afternoon hours}.

Based on the graph charting average ride duration by member type, casual
users have significantly longer ride times compared to annual members.
This makes sense given that casual users are using ride-shares for
leisure versus annual members using them for transit.

\hypertarget{recommendations}{%
\subsection{Recommendations}\label{recommendations}}

Revisiting the business task to determine differences between how casual
users and annual members interact with Cyclistic, I can make a few
recommendations for implementing a strategy to convert memberships.

First, since start hour behavior throughout the week is consistent
between the member types, this suggests that casual members are also
likely to use ride-shares to commute to and from work. These casual
users can be targeted for advertising to encourage membership based on
their frequent use of Cyclistic bicycles.

Further, several stations are shown by descriptive analysis to share
high volume for both casual users and annual members. This can mean
several things. Either these stations are close to businesses with a
high number of employees who commute, they are close to major
recreational areas, or they are along major transit routes. If the
latter is true, specialized advertising can be placed along these routes
explaining the benefits of Cyclistic membership.

If large employers are located near popular stations, it may prove
beneficial to partner with these businesses and offer discounted
Cyclistic memberships for employees who commute. However this would
require further research and effort to build relationships with those
businesses.

\hypertarget{closing-summary}{%
\subsection{Closing Summary}\label{closing-summary}}

For this project, I analyzed a years worth of data for the Cyclistic
bike-share company, located in Chicago. Cyclistic offers a variety of
ride-shares through single ride and day passes (casual users) or annual
memberships. The tools I took advantage of in this instance were Sheets
and RStudio. For my Tableau visualizations and analysis, please {[}visit
here{]}.

The task was to identify how casual users and annual members use
ride-share differently. I hypothesized that casual use was more
recreational in nature, whereas members most likely take advantage of
Cyclistic for commuting. I examined daily, weekly, and monthly usage of
bicycles for each member type. This confirmed my hypothesis that annual
members most likely used the ride-share to commute and casual users
spiked on the weekends. However, casual users mirrored annual member
start hours throughout the week, suggesting that casual users may also
use ride-share to commute.

I suggest that advertisement be aimed at these casual users as they are
the most likely to convert to annual memberships given their frequency
of use. I also suggested that marketing be implemented near high traffic
stations in order to maximize the reach of the campaign. This effort can
be strengthened by further research into the areas surrounding popular
stations and possibly partnering with businesses in the area to offer
discounted Cyclistic membership to employees who commute to work.

\end{document}
